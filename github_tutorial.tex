\documentclass[paper=a4, fontsize=12pt]{scrartcl}
\usepackage[T1]{fontenc}
\usepackage{fourier}
\usepackage{setspace}
\usepackage{wrapfig}
\usepackage{varwidth}% http://ctan.org/pkg/varwidth
\usepackage{subcaption}
\usepackage{geometry}
\geometry{lmargin=1in,rmargin=1in}
\usepackage{alltt}

\usepackage[english]{babel}															% English language/hyphenation
\usepackage[protrusion=true,expansion=true]{microtype}	
\usepackage{amsmath,amsfonts,amsthm} % Math packages
\usepackage[pdftex]{graphicx}	
\usepackage{url}
\usepackage{placeins}

%%% Listing github codes
\usepackage{listings}
\usepackage{xcolor}


%%% Custom sectioning
\usepackage{sectsty}
\allsectionsfont{\small\scshape}

%%% Custom headers/footers (fancyhdr package)
\usepackage{fancyhdr}
\pagestyle{fancyplain}
\fancyhead{}											% No page header
\fancyfoot[L]{\texttt{git help} for help}											% Empty 
\fancyfoot[C]{}											% Empty
\fancyfoot[R]{\thepage}									% Pagenumbering
\renewcommand{\headrulewidth}{0pt}			% Remove header underlines
\renewcommand{\footrulewidth}{0pt}				% Remove footer underlines
\setlength{\headheight}{13.6pt}


%%% Equation and float numbering



%%% Maketitle metadata
\newcommand{\horrule}[1]{\rule{\linewidth}{#1}} 	% Horizontal rule

\title{
		\vspace{-1in} 	
		\usefont{OT1}{bch}{b}{n}
		\normalfont \normalsize \textsc{Github} \\ [25pt]
		\horrule{0.5pt} \\[0.4cm]
		\Large A MINI TUTORIAL \\
		\horrule{2pt} \\[0.5cm]
}
\author{
		\normalfont 								\normalsize
        Aditya Gore\\[-3pt]		\normalsize
        \today
}
\date{}



%\usepackage{Sweave}
\begin{document}
\maketitle
%%% Set github code environment

\lstdefinestyle{gitnumber}{
	numbers=left,
	numberstyle=\tiny,
	backgroundcolor=\color{gray!15},
	upquote=true,
	showstringspaces=false,
	morecomment=[l][\color{blue!80}\ttfamily]{https://},
	morecomment=[l][\color{blue!80}\ttfamily]{http://}	
}

\lstdefinestyle{git}{
	backgroundcolor=\color{gray!15},
	upquote=true,
	showstringspaces=false,
	morecomment=[l][\color{blue!80}\ttfamily]{https://},
	morecomment=[l][\color{blue!80}\ttfamily]{http://}	
}

\lstset{style=git}



\section*{Configure the User Settings}
\begin{enumerate}
\item The commands that are used to configure
\begin{itemize}
\item System wide configuration
\begin{lstlisting}
git config --system
\end{lstlisting}
\item User based configuration
\begin{lstlisting}
git config --global
\end{lstlisting}
\item Project based configuration
\begin{lstlisting}
git config
\end{lstlisting}
\end{itemize}
\item The variables that we can configure are
\begin{itemize}
\item User name
\begin{lstlisting}
git config --global user.name "Your name"
\end{lstlisting}
\item User email address
\begin{lstlisting}
git config --global user.emal "Your email address"
\end{lstlisting}
\item Custom editor
\begin{lstlisting}
git config --global core.editor "Editor"
\end{lstlisting}
\item Git bash colored user interface
\begin{lstlisting}
git config --global color.ui true
\end{lstlisting}
\end{itemize}
\item To view the configurations
\begin{lstlisting}
git config --list
\end{lstlisting}
\end{enumerate}

\section*{Create Repository w/o Github}
\begin{enumerate}
\item If you want to start off with the project without creating a repository on Github then create the directory called ``xyz'' where you want to start the project and then use git bash to navigate to the folder and once in there use the command
\begin{lstlisting}
git init
\end{lstlisting}
This command will start tracking any changes that are made in the ``xyz'' folder from here on.
\end{enumerate}

\section*{Create Repository on Github}
\begin{enumerate}
\item Go to \url{https://github.com/} and create a repository ``xyz''.
\item Once created click on clone or download button and get a url link to the project which will look like \url{https://github.com/yourname/xyz.git}.
\item Open the github bash in folder where you want the project and use the following command to clone the project.
\begin{lstlisting}
git clone https://github.com/yourname/xyz.git
\end{lstlisting}
\item This will create a folder called ``xyz'' in the directory you used the command.
\end{enumerate}

\section*{Making commits}
\begin{enumerate}
\item To add all the changes to the ``staged'' index use the command
\begin{lstlisting}
git add .
\end{lstlisting}
\item To commit the changes to the same ``branch''
\begin{lstlisting}
git commit -m "Write down the message about the commit"
\end{lstlisting}
\item To remove a deleted file (working directory) from the repository
\begin{lstlisting}
git rm file_name
\end{lstlisting}
\item To move or rename a file
\begin{lstlisting}
git mv firstname secondname
\end{lstlisting}
\item To add and commit at the same time
\begin{lstlisting}
git commit -am "write the message here"
\end{lstlisting}
\end{enumerate}

\section*{Viewing commits and logs}
\begin{enumerate}
\item To view commits made so far
\begin{lstlisting}
git log
\end{lstlisting}
\item To view past $3$ commits
\begin{lstlisting}
git log -n 3
\end{lstlisting}
\item We can use time filter to view the commits
\begin{lstlisting}
git log --since=2016-09-15 --until==2016-09-22
\end{lstlisting}
\item We can use author filter to view the commits
\begin{lstlisting}
git log --author="Aditya"
\end{lstlisting}
The author can be matched on partial string but is case sensitive
\item To filter by words in the commit message
\begin{lstlisting}
git log --grep="regexp"
\end{lstlisting}
\item To check the commits from ``HEAD'' going backwards.
\begin{lstlisting}
git log HEAD
\end{lstlisting}
\end{enumerate}

\section*{Managing repositories}
\begin{enumerate}
\item To check the status of the current branch
\begin{lstlisting}
git status
\end{lstlisting}
It gives the difference between the working directory, staging index and git repository.
\item To see the differences between the repository and working directory, file by file
\begin{lstlisting}
git diff filename<optional>
\end{lstlisting}
\item To view the differences between the git repository and staging index, file by file
\begin{lstlisting}
git diff --staged
\end{lstlisting}
\item To view changes side by side
\begin{lstlisting}
git diff --color-words filename
\end{lstlisting}
\item Undo changes in the working directory
\begin{lstlisting}
git checkout -- filename
\end{lstlisting}
This will checkout the the files from the repository at the current branch to the working directory.
\end{enumerate}

\end{document}
